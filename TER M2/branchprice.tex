\subsubsection{Principe général}
Un algorithme de branch and Price est l'équivalent d'un algorithme de Branch and Bound et de génération de colonnes. \\
La différence se fait par incrémentation. A chaque noeud, une résolution de PL. De ce fait on ne garantit pas à chaque noeud une solution optimal ou même valide. 
\\
La première étape d'un branch and price est de générer ce qui s'appele le \textbf{\textit{Master Problem}}.\\
De ce \textbf{\textit{Master Problem}} devront être générés des \textbf{\textit{sous-problèmes}}. \\
Chaque \textbf{\textit{sous-problèmes}} sera résolu et à chaque itération, une résolution de PL sera effectué et chaque solution entrainera une génération de colonne et sera rajouté à la solution. \\
Lorsque tout les \textbf{\textit{sous-problèmes}} générés admettent une solution entièrement négative c'est la fin de l'algorithme.


\subsubsection{exemple}
\begin{center}
\begin{tabular}{ ccc}
\begin{tabular}{  p{2.5cm} c c c}
\hline
\textbf{Job}\textit{(i)} \\demandé \\
\hline
 & 1 & 2 & 3 \\
 \hline
Machine \textit{i} = 1 & 5 & 7 & 3 \\
Machine \textit{i} = 2 & 2 & 10 & 5 \\
\hline
\end{tabular}

\vspace{1cm}
& &
\begin{tabular}{c c}
\hline
\textit{i} & $C_\textit{i}$\\
\hline
1 & 5 \\
\hline
2 & 8\\
\hline
\end{tabular}

\vspace{1cm}

\begin{tabular}{  p{2.5cm} c c c}
\hline
Capacité \textbf{Job}\textit{(i)} \\
\hline
 & 1 & 2 & 3 \\
 \hline
Machine \textit{i} = 1 & 3 & 5 & 2 \\
Machine \textit{i} = 2 & 4 & 3 & 4 \\
\hline
\end{tabular}

\end{tabular}
\end{center}
\vspace{1cm}

Soit \textit{$k_1$} et \textit{$k_2$} l'ensemble des solutions possibles pour les machines 1 et 2 tel que
\\
\textit{$k_1$} = { (1,0,0), (0,1,0), (0,0,1), (1,0,1)} \\
\textit{$k_2$} = {  (1,0,0), (0,1,0), (0,0,1), (1,1,0), (0,1,1)} \\ 

Le \textbf{\textit{Master Problem}} pourra être representé comme ci-suit:
\\
\textbf{max} z = 5$y_1^1$ + 7$y_1^2$ + 3$y_1^3$ + 8$y_1^4$ + 2$y_2^1$+10$y_2^2$ + 5$y_2^3$ + 12$y_2^4$+15$y_2^5$
\\ \\
Voici une décomposition par job: \\
$y_1^1$ + 0 + 0 +$y_1^4$ + $y_2^1$  + 0 + $y_2^4$ + 0 = 1 (\textit{$u_1$}) \\
0 + $y_1^2$ + 0 + 0 + 0 + $y_2^2$ + 0 + $y_2^4$ + $y_2^5$ = 1 (\textit{$u_2$}) \\
0 + 0 + $y_1^3$ + $y_1^4$ + 0 + 0 + $y_2^3$ + 0 + $y_2^5$ = 1     (\textit{$u_3$}) \\

Ici au lieu d'exprimer les \textbf{Machines} par \textbf{Job}, est exprimé les \textbf{Job} par \textbf{Machines}. C'est le Dual.\\
$y_1^1$ + $y_1^2$ + $y_1^3$ + $y_1^4$ $\leq$ 1 (\textit{$v_1$}) \\
$y_2^1$ + $y_2^2$ + $y_2^3$ + $y_2^4$ $\leq$ 1 (\textit{$v_2$}) \\

La prochaine étape est de choisir un ensemble de solution réalisable. Dans notre cas; le choix est relativement facile, mais un domaine d'étude intéressant est l'optimisation de ce choix, trouver rapidement une solution réalisable efficace pour les problèmes plus difficile.
\\
\\
Dans cet exemple, $y_1^1$ et $y_2^5$ seront choisis (arbitrairement). De ce fait voici le nouveau \textbf{\textit{Master Problem restricted}} (RMP) : \\
\textbf{max} \textit{z} = 5$y_1^1$  + 15$y_2^5$ 
\\ \\
\begin{tabular}{c c }
$y_1^1$  + 0 = 1 & \\
0 + $y_2^5$ = 1 & \\
\sout{0 + $y_2^5$ = 1} & \\
\sout{$y_1^1$  + 0 = 1} & \textit{redondant}\\
\sout{0 + $y_2^5$ =1} & \\ 
\end{tabular} 

\vspace{0.5cm}
Voici la matrice de Base B et son inverse: 
\\

\begin{center}
\begin{tabular}{c c}
B=$\begin{bmatrix}
1 & 0 \\
0 & 1
 \end{bmatrix} $                  
&
$B^{-1}$=$\begin{bmatrix}
1 & 0 \\
0 & 1
 \end{bmatrix} $             
\end{tabular}
\end{center}


$y_1^1$ est pour rappel, la variable d'éxécution de la tâche 1 sur la machine 1 son coefficient est de 5. \\
$y_2^5$ est la variable pour le travail sur la machine 2 de la tâche 2 \textbf{et} 3 de ce fait il représente la somme de ces deux coefficients : 10 et 5. \\

De ce fait, pour le moment, la meilleur solution est \textit{$u_1$} = \TextSoulign{red}{ }{ 5},  \textit{$u_2$} = \TextSoulign{red}{ }{15}, \textit{$v_1$} = 0, \textit{$v_2$} = 0.    \\

Nous allons décomposer les sous problèmes par machine:
\begin{itemize}
\item{
Sous problème machine 1 : \\ \\ 
\textit{\underline{Rappel:}}  Tâches demandées pour Machine 1 =  \TextSoulign{blue}{ }{ 5}$x_1^1$ + \TextSoulign{blue}{ }{ 7}$x_1^2$ + \TextSoulign{blue}{ }{3}$x_1^3$  \\ 
Capacité Machine 1 = 3$x_1^1$ + 4$x_1^2$ + $x_1^3$ $\leq$ 5
\\ \\ 
\textbf{max} ( \TextSoulign{blue}{ }{ 5} - \TextSoulign{red}{ }{5} ) $x_1^1$ + (\TextSoulign{blue}{ }{ 7} - \TextSoulign{red}{ }{15} ) $x_1^2$ + (\TextSoulign{blue}{ }{3} - 0) $x_1^3$
\\
 Le coefficient $x_1^2$ étant négatif, il exclut toute solution avec lui dedans et (1, 0, 0) étant nul ce ne sera pas une solution optimal. Quel que soit la solution optimal choisi le max des coefficients sera 3. 
}

\item{
Sous problème machine 2: 
\\
\\
\textit{\underline{Rappel:}}  Tâches demandées pour Machine 2 =  \TextSoulign{green}{ }{2}$x_2^1$ + \TextSoulign{green}{ }{10}$x_2^2$ + \TextSoulign{green}{ }{5}$x_2^3$  \\ 
Capacité Machine 2 = 3$x_2^1$ + 4$x_2^2$ + $x_2^3$ $\leq$ 5
\\ \\ 
\textbf{max} ( \TextSoulign{green}{ }{2} - \TextSoulign{red}{ }{5} ) $x_2^1$ + (\TextSoulign{green}{ }{10} - \TextSoulign{red}{ }{15} ) $x_2^2$ + (\TextSoulign{green}{ }{5} - 0) $x_2^3$
\\
\\ Il existe une solution : (0, 0, 1) car $x_1^2$  et $x_1^1$  étant négatif, ils sont exclus de la solution. Le coefficient maximum possible sera 5.
}
\end{itemize}

Maintenant que les deux problèmes ont été résolu, le \textbf{MPR} va être complété. Des deux solutions le sous problème 2 est celui avec le plus gros coefficient, ce qui représente la plus grosse maximisation.
\\
C'est à dire 5 du sous problème 2 : 5 avec comme solution (0, 0, 1) qui correspond à $y_2^3$. C'est la génération de colonne : une solution vient d'être trouvé et va implémenter notre solution final.
\\ \\ 
Le nouveau \textbf{RMP} passe de :
\begin{center}
 5$y_1^1$  + 15$y_2^5$  \\
 5$y_1^1$  + 15$y_2^5$ +   \TextSoulign{black}{ }{5}$y_2^3$
\end{center}

Voici la nouvelle décomposition :

\begin{tabular}{cc}
$y_1^1$  + 0 + 0 = 1 & \\
0 + $y_2^5$ + 0= 1 & \\
0 + $y_2^5$ + $y_2^3$= 1 & \\
\sout{$y_1^1$  + 0 + 0 = 1} & \textit{redondant}\\
\sout{0 + $y_2^5$ + $y_2^3$ =1} & \\ 
\end{tabular}

\begin{center}
\begin{tabular}{c c c}
B=$\begin{bmatrix}
1 & 0 & 0\\
0 & 1 & 0 \\
0 & 0 & 1 
 \end{bmatrix} $                  
&
$B^{-1}$=$\begin{bmatrix}
1 & 0 & 0\\
0 & 1 & 0 \\
0 & 0 & 1 
 \end{bmatrix} $             
\end{tabular}
\end{center}

De ce fait comme la dernière fois on se retrouve avec une matrice identité et les coefficients restent inchangés. \\
La \textit{$''nouvelle''$} meilleur solution est \textit{$u_1$} = \TextSoulign{red}{ }{ 5},  \textit{$u_2$} = \TextSoulign{red}{ }{15}, \textit{$u_3$} = \TextSoulign{black}{ }{5}, \textit{$v_1$} = 0, \textit{$v_2$} = 0.    
\\
De là, il faudra à nouveau décomposer en sous problème par machine
\\
Nous aurons donc : 
\\
\begin{itemize}
\item{
Sous problème machine 1 : \\ \\ 
\textit{\underline{Rappel:}}  Tâches demandées pour Machine 1 =  \TextSoulign{blue}{ }{ 5}$x_1^1$ + \TextSoulign{blue}{ }{ 7}$x_1^2$ + \TextSoulign{blue}{ }{3}$x_1^3$  \\ 
Capacité Machine 1 = 3$x_1^1$ + 4$x_1^2$ + $x_1^3$ $\leq$ 5
\\ \\ 
\textbf{max} ( \TextSoulign{blue}{ }{ 5} - \TextSoulign{red}{ }{5} ) $x_1^1$ + (\TextSoulign{blue}{ }{ 7} - \TextSoulign{red}{ }{15} ) $x_1^2$ + (\TextSoulign{blue}{ }{3} -  \TextSoulign{black}{ }{5}) $x_1^3$
\\
 Il n'existe plus de solutions optimales tout les coefficients sont nul ou négatifs. 
 \\

}

\item{
Sous problème machine 2: 
\\
\\
\textit{\underline{Rappel:}}  Tâches demandées pour Machine 2 =  \TextSoulign{green}{ }{2}$x_2^1$ + \TextSoulign{green}{ }{10}$x_2^2$ + \TextSoulign{green}{ }{5}$x_2^3$  \\ 
Capacité Machine 2 = 3$x_2^1$ + 4$x_2^2$ + $x_2^3$ $\leq$ 5
\\ \\ 
\textbf{max} ( \TextSoulign{green}{ }{2} - \TextSoulign{red}{ }{5} ) $x_2^1$ + (\TextSoulign{green}{ }{10} - \TextSoulign{red}{ }{15} ) $x_2^2$ + (\TextSoulign{green}{ }{5} - \TextSoulign{black}{ }{5}) $x_2^3$
\\
Encore une fois tout les coefficiens sont nuls ou négatifs, il n y a pas de nouvelles solution.
}
\end{itemize}

Si il n'existe aucune nouvelle solution apporté, il n'existe pas d'autres solutions optimal, elle a déjà été trouvée. La colonne rajoutée n'améliore finalement en rien.
\\
La solution optimal est :
\begin{itemize}
\item{(1, 0, 0) associé à la machine 1}
\item{(0, 1, 1) associé à la machine 2}
\end{itemize}


Voici une représentation du Branch and price:

\begin{figure}[h]
\setlength{
\unitlength}{0.14in}
% selecting unit length
\centering
% used for centering Figure
\begin{picture}(32,50)
% picture environment with the size (dimensions)
% 32 length units wide, and 15 units high.
\put(3,4){
\framebox(6,3){$H_{B}(q)$}}
\put(3,0){
\framebox(6,3){}}
\put(23,4){
\framebox(6,3){$H_{C}(q)$}}
\put(6,4){
\vector(0,-1){1}}
\end{picture}
\caption{An LNL Block Oriented Model Structure}
% title of the Figure
\label{fig:lnlblock}
% label to refer figure in text
\end{figure}







