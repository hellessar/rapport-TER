%%% LaTeX Template
%%% This template can be used for both articles and reports.
%%%
%%% Copyright: http://www.howtotex.com/
%%% Date: February 2011

%%% Preamble
\documentclass[paper=a4, fontsize=11pt]{scrartcl}	% Article class of KOMA-script with 11pt font and a4 format

\setcounter{secnumdepth}{4}
\setcounter{tocdepth}{4}
\makeatletter
\newcounter {subsubsubsection}[subsubsection]
\renewcommand\thesubsubsubsection{\thesubsubsection .\@alph\c@subsubsubsection}
\newcommand\subsubsubsection{\@startsection{subsubsubsection}{4}{\z@}%
                                     {-3.25ex\@plus -1ex \@minus -.2ex}%
                                     {1.5ex \@plus .2ex}%
                                     {\normalfont\normalsize\bfseries}}
\renewcommand\paragraph{\@startsection{paragraph}{5}{\z@}%
                                    {3.25ex \@plus1ex \@minus.2ex}%
                                    {-1em}%
                                    {\normalfont\normalsize\bfseries}}
\renewcommand\subparagraph{\@startsection{subparagraph}{6}{\parindent}%
                                       {3.25ex \@plus1ex \@minus .2ex}%
                                       {-1em}%
                                      {\normalfont\normalsize\bfseries}}
\newcommand*\l@subsubsubsection{\@dottedtocline{4}{10.0em}{4.1em}}
\renewcommand*\l@paragraph{\@dottedtocline{5}{10em}{5em}}
\renewcommand*\l@subparagraph{\@dottedtocline{6}{12em}{6em}}
\newcommand*{\subsubsubsectionmark}[1]{}



\usepackage[english]{babel}														% English language/hyphenation
\usepackage[protrusion=true,expansion=true]{microtype}				% Better typography
\usepackage{amsmath,amsfonts,amsthm}										% Math packages
\usepackage[pdftex]{graphicx}		
\usepackage{color}
% Enable pdflatex
%\usepackage{color,transparent}													% If you use color and/or transparency
\usepackage[hang, small,labelfont=bf,up,textfont=it,up]{caption}	% Custom captions under/above floats
\usepackage{epstopdf}																	% Converts .eps to .pdf
\usepackage{subfig}																		% Subfigures
\usepackage{booktabs}																	% Nicer tables
\usepackage{wrapfig} 
\usepackage{ulem}
%%% Advanced verbatim environment
\usepackage{verbatim}
\usepackage{multicol}
\usepackage{fancyvrb}
\usepackage[utf8]{inputenc}
\DefineShortVerb{\|}								% delimiter to display inline verbatim text


%%% Custom sectioning (sectsty package)
\usepackage{sectsty}								% Custom sectioning (see below)
\allsectionsfont{%									% Change font of al section commands
	\usefont{OT1}{bch}{b}{n}%					% bch-b-n: CharterBT-Bold font
%	\hspace{15pt}%									% Uncomment for indentation
	}

\sectionfont{%										% Change font of \section command
	\usefont{OT1}{bch}{b}{n}%					% bch-b-n: CharterBT-Bold font
	\sectionrule{0pt}{0pt}{-5pt}{0.8pt}%	% Horizontal rule below section
	}


%%% Custom headers/footers (fancyhdr package)
\usepackage{fancyhdr}
\pagestyle{fancyplain}
\fancyhead{}														% No page header
\fancyfoot[C]{\thepage}										% Pagenumbering at center of footer
\fancyfoot[R]{\small \texttt{Master II 2014-2015}}	% You can remove/edit this line 
\renewcommand{\headrulewidth}{0pt}				% Remove header underlines
\renewcommand{\footrulewidth}{0pt}				% Remove footer underlines
\setlength{\headheight}{13.6pt}
\newcommand{\TextSoulign}[3]{\textcolor{#1}{\underline{#2\textcolor{black}{#3}}}}


%%% Equation and float numbering
\numberwithin{equation}{section}															% Equationnumbering: section.eq#
\numberwithin{figure}{section}																% Figurenumbering: section.fig#
\numberwithin{table}{section}																% Tablenumbering: section.tab#


%%% Title	
\title{ \vspace{-1in} 	\usefont{OT1}{bch}{b}{n}
		\huge \strut Recherche rapide d'arbres de Steiner \strut \\
		\Large \bfseries \strut Eric Bourreau \\ Rodolphe Giroudeau \strut
}
\author{ 									
	\usefont{OT1}{bch}{m}{n} David Aubert\\		
	\usefont{OT1}{bch}{m}{n} Deguilhem Julien\\		
	\usefont{OT1}{bch}{m}{n}Université de  Montpellier II\\	
}
\date{20 jan 2015}


%%% Begin document
\begin{document}
\maketitle

\section{Résolution de programmes linéaires}

\subsection{Génération de colonnes}
\subsubsection{Les outils}
La génération de colonnes est une technique utlisée pour résoudre un programme linéaire en nombre réels, lorsque les variables sont trop nombreuses pour être toutes énumérées ou pour une résolution directe par un solveur. La génération de colonnes utilise en autres, la solution duale et les coûts réduits. \cite{introBCBP}
\subsubsubsection{Dualité}
Pour un problème linéaire apellé \textit{Primal}, le Dual est la ''\textit{transposée}'' du \textit{Primal}: les variables du Primal deviennent les constantes du Dual et vice et versa. \\
Le problème de maximisation du primal est le problème de minimisation du dual et réciproquement.
\newline
\\
\underline{\textit{Exemple}} : 
\\


\begin{tabular}{ l p{5 cm}l p{5 cm}  l p{5 cm} l}
Problème Primal : & & 
Problème Dual : \\

$Min{ - x_1 + 4x_2}$ & &  $Max -7\textit{u}_1 + 4\textit{u}_2 + 5\textit{u}_3$\\ 


$\left\lbrace
\begin{array}{l}
-x_1\geq -7\\
3x_1-2x_2 \geq 4\\
x_1+x_2 \geq 5\\
x_1\geq 0, x_2$ libre$\\
\end{array}
\right.$ 
& & 
$\left\lbrace
\begin{array}{l}
-\textit{u}_1 + 3\textit{u}_2 + \textit{u}_3 \leq -1 \\
-2\textit{u}_2 + \textit{u}_3 = 4 \\
u_i \geq 0\\
\end{array}
\right.$
\end{tabular}

\subsubsubsection{Coûts réduits}

Pour résoudre un PL, donné par une matrice de contrainte A il faut une base B. Tel que pour un problème de minimisation il y ait:
\newline
\newline
\hspace{2.5cm}Minimiser $cx$ sous\\
\hspace{2.5cm} A\textit{x = d} \\
\hspace{2.5cm}\textit{ $x \geq 0$}
\\
\\
On réécrit ce problème avec la base B:\\
\hspace{2.5cm}Minimiser $c_B.x_B + c_N.x_N =d$ sous\\
\hspace{2.5cm}$B.x_B + N.x_N =d$\\
\hspace{2.5cm}\textit{ $x_B \geq 0$},\textit{ $x_N \geq 0$}\\

Où $N$ est la matrice qui complete $B$ pour reformer $A$, $x_B$ les variables en base ( Les colonnes choisis dans A pour former B) et  $x_N$ les variables hors-base.

Ici, il est possible de transformer les contraintes d'inégalités en égalités avec l'ajout de variables d'écarts. On les ajoute négativement pour une inégalité "superieur à" et inversement.\\
\\
Pour chaque variables d'écarts, il y a un coût: \textit{$c_j$} \textit{(Coût associé à la variable d'écart j)}.
Son coût réduit est l'écart qu'il y a entre le coût associé à la variable hors-base $x_j$ et les variables duales en base pour la contrainte $j$.\\
$c_j = c_j - c_B.B^{-1}.A_j$\\

Calculer le coût réduit revient à donner "de combien" peut s'améliorer la fonction objective si on relâche la contrainte $j$ d'une unité.
\newline
\\
\underline{\textit{Exemple}} : 
\\
En suivant l'exemple précédent, on y ajoute les variables d'écart $x_3$ $x_4$ et $x_5$ :\\
\begin{tabular}{ p{5 cm}p{3 cm}p{3 cm}p{3 cm}}
 \hline
Problème Primal  & solution & solution duale & coûts réduits \\
 \hline
$Min{ - x_1 + 4x_2}$ & $x_1 = 7$ & $u_1 = 5$ & $c_3 = 5$\\
$-x_1 -x_3 \geq -7$ 
&
$x_2 = -2$
&
$u_2 = 0$
&
$c_4 = 0$\\
$3x_1-2x_2 -x-4 \geq 4$

$x_1+x_2 -x_5 \geq 5$ & $x_4 = 21$ & $u_3 = 4$ & $c_5 = 4$\\
 \hline
\end{tabular}
$x_1$,$x_2$ et $x_4$ sont les variables en base.
\subsubsection{La méthode}
La génération de colonnes fonctionne sur un problème en nombre réels (Pas de PLNE). Elle est utilisé quand le PL possède un très grand nombre de variable. L'idée etant de résoudre le problème maitre en ne prenant en compte qu'un nombre restreint et suffisant de variables.\\

On commence par générer un PL à partir du problème maitre, qu'on va appeler problème maitre restreint (PMR). Le choix de se problème doit se faire de manière heuristique pour optimiser le développement de l'algorithme. Puis on va générer des variables améliorantes susceptibles d'améliorer le résultat.\\

Pour les trouver, on va s'interresser au problème dual et on va regarder les coûts réduits. En effet, le coût des variables duales va nous permettre de trouver une ou plusieurs variables améiorantes.
Rappelons comment obtenir le coût :\\
$c_i = c_i - c_B.B^{-1}.A_i = c_i - u_B.A_i = c_i - sum_{j=1}^{m}{u_j A_{ij}}$\\

avec :\\
$u_i$ la variable duale associée à la contrainte j.\\
$a_{ij}$ le coéfficient de la variable i dans la contraite j.\\
$m$ le nombre de contraintes.\\
\\
Dans un problème de maximisation :\\
la solution optimale s'obtient si tous les coûts réduits sont négatifs. Si ce n'est pas le cas alors on génère de nouvelle(s) variable(s) améliorantes à partir de cette solution pour obtenir un nouveau PMR et recommencer. Les variables ajoutés sont idéalements celles qui n'ont pas encore été insérées dans le problème et qui ont un coût réduit positif.

Dans un problème de minimisation, le problème est le même mais on inverse coût réduit positif et coût réduit négatif.


\subsection{Branch and Price}
\subsubsection{Principe général}
Un algorithme de branch and Price est l'équivalent d'un algorithme de Branch and Bound et de génération de colonnes. \\
La différence se fait par incrémentation. A chaque noeud, une résolution de PL. De ce fait on ne garantit pas à chaque noeud une solution optimal ou même valide. 
\\
La première étape d'un branch and price est de générer ce qui s'appele le \textbf{\textit{Master Problem}}.\\
De ce \textbf{\textit{Master Problem}} devront être générés des \textbf{\textit{sous-problèmes}}. \\
Chaque \textbf{\textit{sous-problèmes}} sera résolu et à chaque itération, une résolution de PL sera effectué et chaque solution entrainera une génération de colonne et sera rajouté à la solution. \\
Lorsque tout les \textbf{\textit{sous-problèmes}} générés admettent une solution entièrement négative c'est la fin de l'algorithme.


\subsubsection{exemple}

\begin{tabular}{  p{2.5cm} c c c}
\hline
\textbf{Job}\textit{(i)} \\
\hline
 & 1 & 2 & 3 \\
 \hline
Machine \textit{i} = 1 & 5 & 7 & 3 \\
Machine \textit{i} = 2 & 2 & 10 & 5 \\
\hline
\end{tabular}

\vspace{1cm}

\begin{tabular}{c c}
\hline
\textit{i} & $C_\textit{i}$\\
\hline
1 & 5 \\
\hline
2 & 8\\
\hline
\end{tabular}

\vspace{1cm}

\begin{tabular}{  p{2.5cm} c c c}
\hline
\textbf{Job}\textit{(i)} \\
\hline
 & 1 & 2 & 3 \\
 \hline
Machine \textit{i} = 1 & 3 & 5 & 2 \\
Machine \textit{i} = 2 & 4 & 3 & 4 \\
\hline
\end{tabular}

\vspace{1cm}

Soit \textit{$k_1$} et \textit{$k_2$} l'ensemble des solutions possibles pour les machines 1 et 2 tel que
\\
\textit{$k_1$} = { (1,0,0), (0,1,0), (0,0,1), (1,0,1)} \\
\textit{$k_2$} = {  (1,0,0), (0,1,0), (0,0,1), (1,1,0), (0,1,1)} \\ 

Le \textbf{\textit{Master Problem}} pourra être representé comme ci-suit:
\\
\textbf{max} z = 5$y_1^1$ + 7$y_1^2$ + 3$y_1^3$ + 8$y_1^4$ + 2$y_2^1$+10$y_2^2$ + 5$y_2^3$ + 12$y_2^4$+15$y_2^5$
\\ \\
Voici une décomposition par job: \\
$y_1^1$ + 0 + 0 +$y_1^4$ + 0 + 0 + $y_2^4$ + 0 = 1 (\textit{$u_1$}) \\
0 + $y_1^2$ + 0 + 0 + 0 + $y_2^2$ + 0 + $y_2^4$ + $y_2^5$ = 1 (\textit{$u_2$}) \\
0 + 0 + $y_1^3$ + $y_1^4$ + 0 + 0 + $y_2^3$ + 0 + $y_2^5$ = 1     (\textit{$u_3$}) \\

Ici au lieu d'exprimer les \textbf{Machines} par \textbf{Job}, est exprimé les \textbf{Job} par \textbf{Machines}. C'est le Dual.\\
$y_1^1$ + $y_1^2$ + $y_1^3$ + $y_1^4$ $\leq$ 1 (\textit{$v_1$}) \\
$y_2^1$ + $y_2^2$ + $y_2^3$ + $y_2^4$ $\leq$ 1 (\textit{$v_2$}) \\

Ici ce sont les contraintes la somme doit être inférieur à 1.
( $\sum_{i=1}^5$
 P(i,j))


\begin{thebibliography}{9}

\bibitem{articleBP}
  Dominique Feillet,
  \emph{Résolution de problèmes de tournées par Branch and Price.}
  Laboratoire d'informatique, Avignon,
  Mars 2008.

\bibitem{introBCBP}
  Hélène Toussaint,
  \emph{Introduction au Branch cut and price et aux solveurs SCIP (Solving constraint integer programs).}
  Rapport de recherche LIMOS /RR-13-07,
  19 avril 2013.

\bibitem{gencol}
  Mads Kemhet Jepsen,
  \emph{Column generation and Branch and Price.}
 
 \bibitem{BrancPriceHuge}
  Mohan Akella, Sharad Gupta, Avijit Sarkar,
  \emph{Branch and Price column generation for solving huge integer programs.}
  Bufallo University (SUNY),

\end{thebibliography}





\iffalse
\section{Génération de colonne}
\begin{hspace}{1cm}
La génération
\end{hspace}
de colonne est une méthode pour résoudre les programmes linéaires de grande taille. Pierre Courteille étudiant le sujet, une collaboration sera effectué. Mais les motivations et les interêts sont grand en effet, les approximations succéssives faites pourront être résolues par cette méthode. Une implémentation du simplex ou par combinaison linéaire pourra être faite.


\section{Recherche et approximation}
\begin{hspace}{1cm}
	Le but de ce TER
\end{hspace}
n'est pas de trouvé la solution optimal mais une solution rapide. Une étude sera faite pour déterminer la k-approximation. Lorsque  nous en avons parlé le k varier entre [2..3].
\newline
Ensuite il faudra se décider entre la programmation linéaire, dynamique et par contrainte. Après une sommaire étude entre ces trois éléments, nous pensons que le plus efficace sera la programmation dynamique (A confirmer).



\section{Initialisation}
\begin{hspace}{1cm}
	Ensuite une 
\end{hspace}
des manières de construire un arbre de Steiner serait en suivant l'exemple de la sélection naturelle. Il faudrait construire un réseau génétique afin de faire hériter à chague génération les meilleurs éléments de la précédente. En effet cela semble être une méthode efficace et naturel. De plus, cela semblerait correspondre au type de programmation choisi, pour le moment dynamique.
\newline
Il faudra pouvoir trouver une heuristique efficace pour générer un ensemble déjà suffisament adapté afin de trouver un nombre de génération d'individu minimal.
\fi
\end{document}
